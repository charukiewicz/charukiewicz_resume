%%%%%%%%%%%%%%%%%%%%%%%%%%%%%%%%%%%%%%%%%
% Medium Length Professional CV
% LaTeX Template
% Version 2.0 (8/5/13)
%
% This template has been downloaded from:
% http://www.LaTeXTemplates.com
%
% Original author:
% Trey Hunner (http://www.treyhunner.com/)
%
% Important note:
% This template requires the resume.cls file to be in the same directory as the
% .tex file. The resume.cls file provides the resume style used for structuring the
% document.
%
%%%%%%%%%%%%%%%%%%%%%%%%%%%%%%%%%%%%%%%%%

% !TEX program = xelatex

%----------------------------------------------------------------------------------------
%	PACKAGES AND OTHER DOCUMENT CONFIGURATIONS
%----------------------------------------------------------------------------------------

\documentclass{resume} % Use the custom resume.cls style

\usepackage[left=0.75in,top=0.6in,right=0.75in,bottom=0.6in]{geometry} % Document margins
\usepackage{fancyhdr}
\usepackage{hyperref}
\hypersetup{
        colorlinks=false,
        linkcolor=none,
        filecolor=none,
        urlcolor=none,
    }

\thispagestyle{fancy}
\fancyhf{} % sets both header and footer to nothing
\renewcommand{\headrulewidth}{0pt}

\name{Christian Charukiewicz} % Your name
\address
    {\href{tel:+16304647644}{630.464.7644}
    $|$ \href{mailto:c.charukiewicz@gmail.com}{c.charukiewicz@gmail.com}
    $|$ \href{https://charukiewi.cz/}{https://charukiewi.cz}
    } % Your phone number and email

\begin{document}

%----------------------------------------------------------------------------------------
%	WORK EXPERIENCE SECTION
%----------------------------------------------------------------------------------------

\begin{rSection}{Professional Experience}

\begin{rSubsection}{Partner \& Principal Software Engineer}{Oct 2020 -- Present}{Foxhound Systems}{}
  \item Led the design, build, and delivery of full stack web applications and back end systems for clients in a variety of industries including ecommerce, video streaming, financial services, and developer tools
  \item Managed software engineers (both in-house and when embeddded in client teams); delegated tasks, reviewed work output and provided constructive feedback, and provided mentorship and career guidance
  \item Worked with product managers \& executives at client companies in problem discovery, requirements gathering
  \item Created software systems using numerous tools and technologies, including Haskell, Node.js, React, Python, PHP, Ruby, Tailwind, MySQL, PostgreSQL
  \item Deployed and maintained client software systems on a variety of platforms, including AWS and Heroku using numerous infrastructure/build tools, such as Docker, Nix, GitHub Actions, BitBucket Pipelines, AWS ECR/ECS/RDS, CloudWatch, Sentry
  \item Co-founding partner; acquired clients, set growth strategy, provided services growing from \$0 to \$500K AR
\end{rSubsection}

%------------------------------------------------

\begin{rSubsection}{Chief Technology Officer}{May 2014 -- Oct 2020}{Roompact}{}
  \item Leader of technology and product driving B2B SaaS revenue growth from \$150K to \$1MM ARR
  \item Set long term road map for product development; continually raised standards for application design, architecture, documentation, performance, security
  \item Led hiring process for technical employees and interns; developed and led training of new technical staff
  \item Introduced and taught the Haskell and Elm programming languages to team members
  \item Planned and executed the transition of company server infrastructure to Amazon Web Services; automated server deployments; reduced technical infrastructure costs by 70\%
  \item Utilized and taught team members on the use of a variety of programming languages, infrastructure tools, and frameworks: Haskell, Elm, JavaScript, CakePHP, Node.js, Jekyll, HTML, SASS/SCSS, Nginx, MySQL, Redis, Supervisor, Ansible, Nix, AWS (EC2, RDS, S3, ELBs, SES, Route 53, Lambda, etc.)
  \item Previous roles: Software Engineer (Oct 2014 - Jan 2016), Software Engineering Intern (May 2014 - Aug 2014)
\end{rSubsection}

%------------------------------------------------

\begin{rSubsection}{Product Development Lead}{Oct 2011 -- Nov 2013}{Next Generation Gaming, LLC}{}
  \item Managed development team consisting of multiple developers working on developing features for small game server with over 8,000 players a month; technologies used: PAWN, MySQL, PHP
  \item Designed and built an automated sales system which increased average monthly revenue by 35\%
\end{rSubsection}

\end{rSection}

%----------------------------------------------------------------------------------------
%	TECHNICAL STRENGTHS SECTION
%----------------------------------------------------------------------------------------

\begin{rSection}{Technical Skills \& Languages}

\begin{tabular}{@{} >{}l @{\hspace{2ex}} l}
    Haskell, Elm, PHP, JavaScript, Python, Ruby, Node.js, Bash, HTML, SCSS, Linux, Docker, Nix, Redis, \\
    Ruby on Rails, React.js, Laravel, Nginx, Apache, Ansible, MySQL, PostgreSQL, SQLite, Git, AWS, Heroku \\
\end{tabular}

\end{rSection}

%----------------------------------------------------------------------------------------
%	EDUCATION SECTION
%----------------------------------------------------------------------------------------

\begin{rSection}{Education}
{\bf University of Illinois, Urbana-Champaign} \hfill {\textsc{2010 -- 2014}} \\
B.Sc. Mathematics \& Computer Science, Philosophy
\end{rSection}

%----------------------------------------------------------------------------------------
%	OSS SECTION
%----------------------------------------------------------------------------------------

\begin{rSection}{Open Source Software}

\begin{rListSection}
\item \textbf{\texttt{isbn}} {\em (Haskell package, Author)} -- \href{https://hackage.haskell.org/package/isbn}{hackage.haskell.org/package/isbn}, \href{https://github.com/charukiewicz/hs-isbn}{github.com/charukiewicz/hs-isbn}
    \begin{itemize} \itemsep -0.5em \vspace{-0.5em}
            \item[-] Authored Haskell library ISBN validation \& manipulation; wrote documentation, automated tests
    \end{itemize}
\item \textbf{\texttt{yesod-auth}} {\em (Haskell package, Contributor)} -- \href{https://hackage.haskell.org/package/yesod-auth}{hackage.haskell.org/package/yesod-auth}
    \begin{itemize} \itemsep -0.5em \vspace{-0.5em}
    \item[-] Enhanced the developer-facing API for email-based user registration and password reset workflows
    \end{itemize}
\item \textbf{\texttt{esqueleto}} {\em (Haskell package, Contributor)} -- \href{https://hackage.haskell.org/package/esqueleto}{hackage.haskell.org/package/esqueleto}
    \begin{itemize} \itemsep -0.5em \vspace{-0.5em}
    \item[-] Ported elements of standard SQL into the package, improving its coverage of SQL
    \item[-] Wrote detailed documentation and examples for an experimental new syntax in the package
    \end{itemize}
\end{rListSection}

\end{rSection}

%----------------------------------------------------------------------------------------
%	EXAMPLE SECTION
%----------------------------------------------------------------------------------------

%\begin{rSection}{Section Name}

%Section content\ldots

%\end{rSection}

%----------------------------------------------------------------------------------------

\cfoot{\footnotesize{\textsc{Revision: March 2023}}}

\end{document}
