%%%%%%%%%%%%%%%%%%%%%%%%%%%%%%%%%%%%%%%%%
% Medium Length Professional CV
% LaTeX Template
% Version 2.0 (8/5/13)
%
% This template has been downloaded from:
% http://www.LaTeXTemplates.com
%
% Original author:
% Trey Hunner (http://www.treyhunner.com/)
%
% Important note:
% This template requires the resume.cls file to be in the same directory as the
% .tex file. The resume.cls file provides the resume style used for structuring the
% document.
%
%%%%%%%%%%%%%%%%%%%%%%%%%%%%%%%%%%%%%%%%%

% !TEX program = xelatex

%----------------------------------------------------------------------------------------
%	PACKAGES AND OTHER DOCUMENT CONFIGURATIONS
%----------------------------------------------------------------------------------------

\documentclass{resume} % Use the custom resume.cls style

\usepackage[left=0.75in,top=0.6in,right=0.75in,bottom=0.6in]{geometry} % Document margins
\usepackage{fancyhdr}
\usepackage{hyperref}
\hypersetup{
    colorlinks=false,
    linkcolor=none,
    filecolor=none,
    urlcolor=none,
}

\thispagestyle{fancy}
\fancyhf{} % sets both header and footer to nothing
\renewcommand{\headrulewidth}{0pt}

\name{Christian Charukiewicz} % Your name
\address
    {\href{tel:+16304647644}{630.464.7644}
    $|$ \href{mailto:c.charukiewicz@gmail.com}{c.charukiewicz@gmail.com}
    $|$ \href{https://charukiewi.cz/}{https://charukiewi.cz}
    } % Your phone number and email

\begin{document}

%----------------------------------------------------------------------------------------
%	WORK EXPERIENCE SECTION
%----------------------------------------------------------------------------------------

\begin{rSection}{Professional Experience}
\begin{rSubsection}{Chief Technology Officer}{2014 -- Present}{Roompact}{}
\item Set long term road map for product development; continually raised standards for application design, architecture, documentation, performance, security
\item Led hiring process for technical employees and interns; developed and led training of new technical staff
\item Chief architect of new features, functionality, and underlying infrastructure; led team members in making design decisions ranging from database schemas,  to user interfaces, to application-specific REST APIs
\item Introduced the Haskell and Elm programming languages as the foundation for new development; taught Haskell, Elm, and functional programming concepts to technical team members with no previous FP experience
\item Planned and executed the transition of company server infrastructure to Amazon Web Services; reduced technical infrastructure costs by 70\%
\item Employed a variety of AWS tools and services, including EC2, RDS, S3, ELBs, SES, Route 53, Lambda, CloudWatch, CloudFront, IAM, Inspector, etc.
\item Achieved completely automated and reproducible Linux server deployments and configuration management with Bash, Ansible, and Nix
\item Utilized and taught team members on the use of a variety of programming languages, infrastructure tools, and frameworks: Haskell, Elm, JavaScript, CakePHP, Node.js, Jekyll, HTML, SASS/SCSS, Nginx, MySQL, Redis, Supervisor, Ansible, Nix
\end{rSubsection}

%------------------------------------------------

\begin{rSubsection}{EMT-Basic (Volunteer Position)}{2012 -- 2014}{Illini Emergency Medical Services}{}
\item Worked with and led teams of EMS providers at designated events
\item Provided pre-hospital basic life support to patients
\end{rSubsection}

%------------------------------------------------

\begin{rSubsection}{Director of Community Experience}{2012 -- 2013}{Next Generation Gaming, LLC}{}
\item Managed development team consisting of multiple developers working on developing features for small game server with over 8,000 players a month
\item Designed and built an automated sales system which increased average monthly revenue by 35\%
\item Worked with the PAWN language, MySQL, and PHP
\end{rSubsection}

\end{rSection}

%----------------------------------------------------------------------------------------
%	OSS SECTION
%----------------------------------------------------------------------------------------

\begin{rSection}{Open Source Software}

\begin{rListSection}
\item \textbf{\texttt{isbn}} {\em (Author)} -- \href{https://hackage.haskell.org/package/isbn}{hackage.haskell.org/package/isbn}, \href{https://github.com/charukiewicz/hs-isbn}{github.com/charukiewicz/hs-isbn}
    \begin{itemize} \itemsep -0.5em \vspace{-0.5em}
    \item[-] Authored the \texttt{isbn} Haskell library, which enables validating and manipulating ISBNs.
    \item[-] Wrote detailed documentation, use-case examples, and automated tests.
    \end{itemize}
\item \textbf{\texttt{yesod-auth}} {\em (Contributor)} -- \href{https://hackage.haskell.org/package/yesod-auth}{hackage.haskell.org/package/yesod-auth}
    \begin{itemize} \itemsep -0.5em \vspace{-0.5em}
    \item[-] Made improvements to the \texttt{yesod-auth} Haskell package, enhancing the developer-facing API to enable more control over email-based user registration and password reset workflows
    \end{itemize}
\item \textbf{\texttt{esqueleto}} {\em (Contributor)} -- \href{https://hackage.haskell.org/package/esqueleto}{hackage.haskell.org/package/esqueleto}
    \begin{itemize} \itemsep -0.5em \vspace{-0.5em}
    \item[-] Ported elements of standard SQL into the \texttt{esqueleto} Haskell package, improving its coverage of SQL
    \item[-] Wrote detailed documentation and examples for an experimental new syntax in the package, demonstrating usage of new package features and explaining design decisions
    \end{itemize}
\end{rListSection}

\end{rSection}

%----------------------------------------------------------------------------------------
%	TECHNICAL STRENGTHS SECTION
%----------------------------------------------------------------------------------------

\begin{rSection}{Technical Skills \& Languages}

\begin{tabular}{@{} >{\bfseries}l @{\hspace{2ex}} l}
Languages
    & Haskell, Elm, PHP, JavaScript, SQL, Python, Bash, HTML, SCSS \\
Technologies
    & Linux, Nix, Redis, Nginx, Ansible, Node.js, MySQL, PostgreSQL, SQLite, Git, AWS \\
\end{tabular}

\end{rSection}

\cfoot{\footnotesize{\textsc{Revision: May 2020}}}

%----------------------------------------------------------------------------------------
%	EDUCATION SECTION
%----------------------------------------------------------------------------------------

\begin{rSection}{Education}
{\bf University of Illinois, Urbana-Champaign} \hfill {\textsc{2010 -- 2014}} \\ 
B.Sc. Mathematics \& Computer Science, Philosophy
\end{rSection}

%----------------------------------------------------------------------------------------
%	EXAMPLE SECTION
%----------------------------------------------------------------------------------------

%\begin{rSection}{Section Name}

%Section content\ldots

%\end{rSection}

%----------------------------------------------------------------------------------------

\end{document}
